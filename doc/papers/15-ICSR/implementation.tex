\vspace{-0.2cm}
\section{Examples of personalized interactions}
\vspace{-0.1cm}

In this section we will show through a set of examples how general purpose social norms are used to affect the behavior of the robot in a declarative way. The examples are taken from the use cases of the {\coaches} project and they will be eventually fully implemented and tested with real robots in the shopping mall in Caen. The examples below refer to the social norms described in Section \ref{sec:pers} and assume user profiles are available.

\paragraph{Example 1. Advertising.} One of the tasks of the {\coaches} robot is to show advertisements to users of the shopping mall. These advertisements (in forms of text, images, videos, etc.) are provided by the shop manager and stored in the Media Library. 
In one form of advertising planned in the project the robot knows the user profile. In this case the Interaction Module described in the previous section can activate personalized messages.
Effects of personalized interactions in this example are: i) animation instead of videos for children, ii) big fonts and simple GUI for elderly people, etc. 

\paragraph{Example 2. Directions and guiding.} The robot is able to give directions and guide people in the mall. Requests are acquired either by voice or graphical interface and the robot uses its semantic map of the environment to show directions or accompany the user. In this case the following personalized behaviors can be obtained: i) for elderly people, a simple GUI shows the direction; ii) the interaction with a deaf and elder person is made with graphical interface only; iii) the interaction with a blind person uses only voice. In all the three cases, the robot offers to accompany them and for the blind person a special notification is given with instructions of how the guidance will happen.

\paragraph{Example 3. Baby care rooms.} Baby care rooms can be used by parents, but must be reserved and they are locked when not in use. The robot can enable this service upon request. Some personalized interactions in this case are: i) a new user is fully instructed with detailed instructions about how to use the service; ii) a user that already used the service a few time ago is given directly the access to the baby care room; iii) children or very young users will be notified that they are not allowed to use the service.

\vspace{1em}

Notice that all these examples are implemented without explicit coding the corresponding behaviors. The expected personalized behavior is the effect of the application of the social norms to the user profile and of the corresponding modifications of the plans that activate actions with proper parameters. Notice also that the social norms are not specific of any particular task. This allows for a high level of extendibility. For example, adding, removing or modifying social norms allow for a significant change of behavior of the robot with different users without requiring any change (or just minimal changes) in the implementation of the actions. For example, assuming that we want to add the capability of the robot to regulate the volume of its voice and to personalize this feature. With our architecture it is sufficient to do the following steps: add a parameter about volume in the \emph{Say} action (e.g., corresponding to a new proposition loud\_speech in $I$) and a social norm 
(elder, loud\_speech)  in $S$. All the interactions with elder people now will use an increased volume of the robot speech.

