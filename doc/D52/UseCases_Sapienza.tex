\section{Introduction}

The COACHES project aims at developing a prototype robotic system for user assistance in a shopping center. In this document we describe a set of use cases that will be implemented to show the feasibility of the approach and to assess the effectiveness of the proposed solutions with actual users.

More specifically, in this document we describe a set of uses cases that focus on two main components of the entire system: 1) short term human-robot interaction, 2) safe navigation.


\section{Use cases for short-term interaction and navigation}


\subsection{Customer asks for help carrying his/her bag}

The robot is called by a manager (or by the customer himself) to assist
someone in carrying his/her bag. The robot must reach
the exit of the shop and approach the dedicated loading area. When
in position he looks for the customer and as soon as he establishes
contact, he asks the user to load the bag in the appropriate container.
Once loaded, the robot will ask the customer if he is in a hurry.
If the customer is in a hurry, the robot will proceed at a sustained
speed to the parking lot. In the other case, the robot will leisurely
proceed to the parking lot, proposing intermediate stops to shops
with interesting discounts.



\subsection{Proposing an ice cream to a child}

Manager from local ice cream shop asks robot to inform children about
a special discount. The robot looks around for children. When it detects
one, it approaches him and informs him and his parents of the discount. The child
is passionate about having an ice cream; but before directing him
there, the robot asks for the parent's authorization. Once the parents
agree to allow the child to follow the robot, the robot will proceed
to the ice cream shop. 


\subsection{Customer refuses first robot proposal, accepts specific one}

Managers have told the robot that he needs to promote a special discount
for a movie. The robot is also aware, in his persistent KB, of the
existence of other secondary discounts. Since there are no people
moving around the robot starts waiting and scanning. As soon as it
sees a suitable user (at the right distance and the right speed),
the robot intercepts the customer and asks him/her if he would be interested
in the special discount for the movie. The user answers that s/he is not interested.
The robot then asks the user to slide his customer card in order to
propose the most appropriate available discount. After reading the
card, the robot finds out that the customer is a woman and that she
has made many purchases in the personal care department. 
So, the robot reasons that the special discount on the facial cream would be of interest
to her. After being informed of the discount, the customer reveals
to be interested and asks the robot for directions to reach the specific
shop. Since not many people are moving around, the robot can safely
guide the customer himself.


\subsection{Customer asks for directions in crowded environment}

The robot has the objective to inform the customers of a certain few
available discounts. The shopping mall is very crowded and, for safety
considerations, the robot decides it should not move but rather wait
for users to approach. As soon as it detects a user, standing still at
the appropriate social distance, looking at him/her, the robot initiates
conversation and announces the available offers. The user is interested
and asks for directions. Since the robot cannot safely move, it shows
the desired path on his tablet and informs the user that 20m along
the corridor s/he will find another robot for more specific information,
if needed. The human thanks the robot and goes to the next one to
receive further directions.
