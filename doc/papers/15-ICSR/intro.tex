\section{Introduction}

Public spaces in large cities are increasingly becoming complex and unwelcoming environments because of the overcrowding and complex information in signboards. It is in the interest of cities to make their public spaces easier to use, friendlier to visitors and safer to increasing elderly population and to the people with disabilities.

In the last decade, we observe tremendous progress in the development of robots in dynamic environments %with presence of
including people. The new challenge for the near future is that robots will be deployed in public areas (malls, touristic sites, parks, etc.) to offer services to make the use of the environment welcoming and easy to use by visitors, elderly or disabled people. 

Such application domains require robots with new capabilities leading to new scientific challenges: robots should assess the situation, estimate the needs of people, socially interact in a dynamic way and in a short time, with many people, the navigation should be safe and respects the social norms. These capabilities require new skills including robust and safe navigation, robust image and video processing, short-term human-robot interaction models, human need estimation techniques and distributed and scalable multi-agent planning.


