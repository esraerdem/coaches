\section{Related Work}


service robots interacting daily with people ...

\cite{gockley2005} shows after the first contact with the robot which
usually takes more time, the time of the interactions decreases after
the novelty effect, suggesting people's preference for short-terms
interactions. 

\cite{Lee2012} shows in a 4-month experiment that personalized
interactions allow to maintain the interest of the users over time.
The experiment consisted of a robot delivering snacks in a workplace
and the personalization was carried out through customized dialogues where the
robot addressed the users by their names and commented the users'
behaviours like their frequency of usage of the service or their snack choice patterns.

Conversely, in certain contexts like in rehabilitation robotics, it is
desired to have longer interactions with the patient, so the robot can
assist and encourage him to do his exercises. In \cite{tapus2008}, it
is shown how adapting the robot behaviour to the patient personality
(introvert or extrovert) increases the level of engagement in the
interaction.



\cite{mason2011robot} User commands tasks to robot which learns user's preferences
registered in a user profile where the world states are defined as
good or bad. With this knowledge and the feedback it receives from the user, the robot can anticipate his needs and
pro-actively act to fullfil his preferences.

\cite{mitsunaga2008adapting} the robot adapts its behaviour defined by
parameters like the distance to the person or its motion speed among
others using a reinforcement learning technique where feedback from
the user is given subsconciously and is read from the robot sensors.
