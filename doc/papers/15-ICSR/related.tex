\section{Related Work}
\vspace{-0.1cm}

The use of service robots interacting daily with people in public
spaces or workplaces has become of increased interest in the last
years.
In this context, the development of the robotic system should focus on
creating confortable interactions with the people the robot has to
share its space.

Gockley \textit{et al.}~\cite{gockley2005} showed that people usually
express more interest and spend more time during the first contact
with the robot. However, after the novelty effect, the time of the
interactions decreases which suggests people's preference for
\textit{short-term} interactions.

In order to address this decrease in the people engagement, Lee
\textit{et al.}~\cite{Lee2012} demonstrated in a 4-month experiment
that \textit{personalized} interactions allow to maintain the interest of the
users over time.  The experiment consisted of a robot delivering
snacks in a workplace and the personalization was carried out through
customized dialogues where the robot addressed the users by their
names and commented the users' behaviours like their frequency of
usage of the service or their snack choice patterns.
Conversely, in certain contexts like in rehabilitation robotics, it is
desired to have longer interactions with the patient, so the robot can
assist and encourage him to do his exercises. In \cite{tapus2008}, it
is shown how adapting the robot behaviour to the patient personality
(introvert or extrovert) increases the level of engagement in the
interaction.

Certain works aim at personalizing the interaction by learning from its
user. For example, in \cite{mason2011robot} a certain task is
commanded to the robot which receives a feedback from the user if the
final state of an action is desirable for him.  In this way, the robot
learns from the user's preferences which are registered in a user
profile. With this knowledge and the feedback it keeps receiving from
the user, the robot can anticipate his needs and pro-actively act to
fullfil his needs.  In \cite{mitsunaga2008adapting} the robot adapts
its behaviour defined by parameters like the distance to the person or
its motion speed, among others, using a reinforcement learning technique
where feedback from the user is given subsconciously through body
signals read directly from the robot sensors.

In contrast to these works, our approach for personalized human-robot
interaction is not based on learning the personality of the user, but
on a set of \textit{social norms} that are present in our everyday
lifes in human interactions.  Moreover, our architecture is designed on domain and task-independent representation of information, providing for a high variability of personalized behaviors, with a simple declarative definition of the social norms that we want the robot to apply. This provides many advantages in terms of extendibility and scalability of the system.
The proposed approach extends a previous work \cite{Nardi14} about the design of social plans, by adding the notions of user profiles and of personalized interactions.


