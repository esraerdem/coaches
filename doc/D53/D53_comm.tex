\section{Communication Infrastructure}

%\subsection{Multi-robot communication}

In this section, we describe the communication infrastructure between the COACHES robots and between a robot and the other components of the system (e.g., fixed cameras in the environment, management tools, etc.)

A multi-robot system can be seen as a set of agents that cooperate together to accomplish a global task. %Each agent is a software component that can be decomposed in different modules each of which responsible for a specific task, such as task planning, state estimation, data acquisition, human-robot interaction, etc. 
One of the greatest challenges in such systems is to provide a stable and reliable infrastructure which allows the communication among specific modules of each agent.

We can divide the problem of communication in two different cases: \emph{local data transmission} between  modules running on the same physical machine, and \emph{remote communication} where modules are distributed within the network.
With modern operating systems, both problems can be solved by mapping each individual module as a software process, transforming the task of information exchange in the well-known problem of inter-process communication. In this way each process receives a separate address space, allowing a physical and semantic separation between modules. Each of them can be written in a different language and can run on different or same host devices. There are also some stability benefits. In fact, if a module has some failures, it does not necessarily impact others. 

For exchanging information between modules, we have to deal with different issues: what kind of information have to be passed, an efficient system to encode them, a way to send messages from one module to another, and a system for decoding messages.

The following features are common in all communication systems:

\begin{itemize}
	
	\item[A] Type Specification
	
	To specify the semantics of the messages, it is necessary to define a representation (or data structure) for each message.
	This representation allows for describing all the information that are exchanged in the system.
    The definition has to be independent from the programming languages, and some tools for read data from messages have to provided.
	
	\item[B] Encoding/Decoding
	
	Messages to be exchanged have to be encoded in a binary format, and all the modules must know how to interpret the binary contents of a message. This process must be as efficient as possible, to not affect the performance of the entire system. 
	
	
	\item[C] Reliability
	
	Due to the possibility of errors during data transmission, it is necessary to include within the encoded message, some process of integrity verification, like a checksum.
	
	\item[D] Communication

	In most multi-robot systems, the transmission and reception of messages lean on standard communication protocols, typically TCP or UDP. 
	
	This part can be \emph{centralized}, with one single node that keeps track of all the modules and all the agents dispatching messages to the right receiver, or can be \emph{distributed} so all the agents know the other participants and can send the information directly to interested. 
	A communication model widely used is the publish/subscribe model, in which some communication channels are defined in advanced  and all the modules can publish or listen to this. Usually it requires a mediator that dispatches directly the message from publisher to all the subscribers or it is used to broker point-to-point connection.

\end{itemize}


\subsection{Lightweight Communications and Marshalling (LCM)}

Lightweight Communications and Marshalling (LCM)\footnote{https://lcm-proj.github.io/} \cite{lcm-iros2010} is a message passing system for interprocess communication that is specifically targeted for the development of real-time systems. LCM provides tools for marshalling, communication, and analysis. It uses a “push”-based publish/subscribe model using UDP multicast protocol as a low-latency but unreliable transport, thus avoiding the need for a centralized hub. LCM provides tools for generating marshalling code based on a formal type declaration language; this code can be generated for a large number of platforms and operating systems and provides run-time type safety.

The following features are relevant in LCM.
	\begin{itemize}
	
	\item[A] Type Specification
	
	A platform independent language for describing messages is defined, and an automatic code generating tool is provided.
	
	\item[B] Marshalling
	
	An efficient encoding/decoding tool is provided for converting messages to a binary format and vice versa. 
	
	\item[C] Fingerprint 
	
	Each message has a fingerprint that depends on its type, and when a message is received if the fingerprint does not match, an error is reported.
	
	\item[D] Communication
	
	LCM uses a publish/subscribe model led on UDP Multicast layer. Using this standard communication protocol, each message is broadcast to all local and remote modules, and a client simply discards messages to which it is not subscribed. In this way the network load grows up, but there is no need of a centralized hub and consequently there is no single failure point or bottleneck in the system.

	\end{itemize}


\subsection{ROS Communication}

Robot Operating System (ROS)\footnote{www.ros.org} is the most common robot middleware used in many robotics applications.
At the lowest level of its architecture, ROS offers a message passing interface that provides inter-process communication.

The ROS middleware offers these facilities:
	\begin{itemize}
	\item[A] Type Specification
		
	The structure of each message interface is defined using a specific IDL (Interface Description Language). A message generator that translate the IDL files into source code is provided.
	
	\item[B] Encoding/Decoding
		
    The conversion in binary format for complex data is automatic and platform independent.
		
		
	\item[C] Reliability
	
	Each message type has an MD5 sum of the IDL file. Modules can only communicate messages for which both the message type and MD5 sum match.

	\item[D] Communication

	ROS built-in messaging infrastructure manages the communication between distributed nodes via the anonymous publish/subscribe mechanism lean on a TCP layer. This standard protocol, guarantee reliability, ordered receive and error checked delivery for all the messages.
	
	\end{itemize}
	
	
 
	%The roscore works as centralized hub to manage all the messages impairing the communication between different physical machines. 

\subsection{TCP Interface}

\begin{figure}
\centering
\includegraphics[width=0.8\linewidth]{fig/Tcp-Interface.pdf}\caption{tcp\_interface architecture.}
\label{fig:tcp_interface}
\end{figure}


As already mentioned, a multi-robot system depends on local message exchange (needed by the different modules composing each agent) and remote message passing (required for coordination among the agents). The local communication can be efficiently managed using the ROS messaging system, but its centralized approach requires a unique roscore running, impairing the communication between different physical machines.
On the contrary, LCM can be used to cope with both local and remote data transmission, provided reliability is not a desired feature, since UDP protocol does not guarantee the correct exchange of data, in terms of error correction and packet delivery.

To overcome the limitations of ROS built-in system and LCM, we built a higher-level block on top of ROS middleware: a ROS node called tcp\_interface\footnote{https://github.com/gennari/tcp\_interface}.

The tcp\_interface node allows for a peer-to-peer communication between different agents that are on the same network. Figure \ref{fig:tcp_interface} shows an architectural diagram of the node that consists of three different layers:
	\begin{itemize}
		\item Discovery layer:
			Using UDP protocol, every agent periodically broadcasts a heartbeat message on the network. This mechanism allows each agent to maintain an updated list of all live peers.
		\item Communication layer:
			It consists of a server that is responsible for receiving messages from other agents, and a client that deals with sending messages to other peers. This layer use the TCP protocol to guarantee the correctness of the packet delivery.
		\item ROS-Bridge:
			Based on the ROS publish/subscribe model, this layer is responsible of collecting the messages generated by each local module and to wrap them in a message compliant with the communication layer.
	
	\end{itemize}


By using this architecture, string messages are sent by each node through a ROS standard communication channel (typically /RCOMMessage) as local data. In such a way, it is possible both to exploit at maximum all the ROS communication features and to obtain a distributed mechanism for exchanging messages, like LCM. 
	
This node is also used for communication between ROS nodes and non-ROS software components, by simply using the TCP channel.
In this way TCP string messages can be easily excahnged between a ROS node (through its standard ROS publish/subscribe communication channel) and a non-ROS software through standard TCP connection (e.g., telnet).
External non-ROS program can be written in any language and run in any operating system on a different machine from the one running ROS.
This mechanism is used in the COACHES architecture to connect the ROS components controlling the robot and running on a Linux laptop with the user interfaces running on the tablet of the robot under the Microsoft Windows operating system.









